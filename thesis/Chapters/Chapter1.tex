\chapter{Увод}

\label{Chapter1}

%----------------------------------------------------------------------------------------

\section{Актуалност на проблема и мотивация}
Бързо разрастващата се сфера на мигновените съобщения (instant-messaging) поражда голям интерес за създаване на интелигентни системи за водене на диалог, още известни като чатботове.\\
Изграждането на такова приложение е трудна задача, поради липсата на структура в текста. Трудността се засилва и от факта, че при чат приложенията правописът и граматиката не са приоритет за потребителите. При зададен въпрос от потребител, системата трябва да генерира смислен и релевантен отговор. \\
Подобна задача се решава при машинния превод. При дадена поредица от думи, трябва да се генерира друга поредица от думи. Целта на тази дипломна работа е да се адаптира модел от машинния превод за изграждане на чатбот.



\section{Цел и задачи на дипломната работа}
Основната цел на тази работа е изследване на възможностите за изграждане на диалогова система базирана на модела seq2seq. Този модел е в основата на съвременните системи за машинен превод, използващи изкуствени невронни мрежи. Ще се изследват и възможностите за интегриране на модела в система за автоматично отговаряне на въпроси.

Задачи, произтичащи от целта:
\begin{itemize}
	\item Обзор на подходите за решаване на този вид задачи
	\item Набавяне на набор от данни - Ubuntu Dataset Corpus и QatarLiving
	\item Предварителна обработка на данните
	\item Имплементация и интеграция на seq2seq модела
	\item Анализ на процеса на машинно самообучение
	\item Тестване, настройка и оценяване на работата на системата
	\item Анализ на резултатите
	\item Интеграция в системата за автоматично отговаряне на въпроси
	\item Разработване на интерактивно демонстративно приложение
\end{itemize}


\section{Структура на дипломната работа}


