\chapter{Преглед на предметната област} % Main chapter title

\label{Chapter2} 

%----------------------------------------------------------------------------------------

\section{Въведение}
Изграждането на отворена диалогова система е предизвикателна задача, с която са се сблъсквали много екипи. През годините са опитвани различни подходи за изграждане на подобна система и в тази глава ще се запознаем с основните методи и тяхната еволюция. 
Благодарение на прогреса в сферата на машинното самообучение и на нарастващата изчислителна мощ бяха постигнати подобрения в задачата за автоматичен машинен превод. Ще обобщим тяхната архитектура и ще коментираме приложимостта на подобен модел за изграждане на диалогова система.
Ще разгледаме и проблема за отговаряне на често задавани въпроси и връзката му с диалоговите системи


https://research.fb.com/the-long-game-towards-understanding-dialog/

\section{Диалогови системи: мотивация и история}

-мотивация
-ELIZA
-PARRY?
-ALICE
-games
-assistants

\section{Машинен превод}
- statistical machine translation
- neural machine translation
- seq2seq


\section{Отговаряне на често задавани въпроси}
